% sage_latex_guidelines.tex V1.20, 14 January 2017

\documentclass[sagev,times,Review]{sagej}
% add 'doublespace'

\usepackage{moreverb,url}

\usepackage[colorlinks,bookmarksopen,bookmarksnumbered,citecolor=red,urlcolor=red]{hyperref}

\newcommand\BibTeX{{\rmfamily B\kern-.05em \textsc{i\kern-.025em b}\kern-.08em
T\kern-.1667em\lower.7ex\hbox{E}\kern-.125emX}}

\def\volumeyear{2021}

\begin{document}

\runninghead{Pape}

\title{Energy simulation and analysis of accessory dwelling unit: 
an evolutionary approach}

\author{Preston Pape\affilnum{1}}

\affiliation{\affilnum{1}University of Washington}

%\corrauth{}

\email{ppape@uw.edu}

\begin{abstract}
The United States is experiencing an unprecedented housing crisis, resulting in a rising population of unhoused peoples and the inability to become a homeowner or to even afford monthly rent in many cities. Solutions to this dilemma are neither straightforward nor definite. Seattle, Washington; Portland, Oregon; and Vancouver, British Columbia are exploring the use of accessory dwelling units (ADUs) as one method to combat surging housing prices. Use of ADUs is an effective means of increasing housing density without replacing single family housing zones with new multi family residential construction. Additionally, ADUs are often designed to be rented, generating supplementary income for the homeowner. Ten detached ADU (or DADU) designs are pre-approved by the City of Seattle and are freely available online to entice homeowners. However, a 2019 city survey shows that there are calls for an increased focus on sustainability and cost. This research intends to explore whether the use of genetic algorithms via shape grammar methodology to optimize DADU plans to site context, increases building performance or further encourages construction. The proposed methodology begins by reading example site data from the city of Seattle including building and vegetation context and rental/land prices from Seattle GIS and Zillow, respectively. Next, a genetic algorithm explores the design space for a viable floor plan solution based on a fitness function. This fitness function evaluates individual designs according to predefined traits. Traits to evaluate include window to wall ratio, insulation depth/type, ventilation strategy, and shading technique. Locating the correct combination of traits to minimize (or maximize) which results in a higher performance DADU is the desired outcome. Resulting designs will be analyzed and compared via energy performance simulation. The end-goal is to develop a computational tool using the aforementioned system to conduct automated site analysis, as well as parametric generation of DADUs.\end{abstract}

\keywords{Genetic Algorithm, ADU, Optimization, Simulation, EUI, Grasshopper, Galapagos}

\maketitle

\section{Introduction}
The United States is amid an unprecedented housing crisis, stemming from issues outside of housing, such as neoliberal cuts to social benefits spending and the increasing privatization of essentially every aspect of American life. The result is a rising population of unhoused peoples and the inability for many to become a homeowner or to even afford minimum monthly rent in many cities. Solutions to this dilemma are neither straightforward nor definite. However, as designers and architects, this problem can only be addressed at the symptom-level, while advocating at the root cause. 

%TODO Expand background portion of intro

Seattle, Washington; Portland, Oregon; and Vancouver, British Columbia are exploring the use of accessory dwelling units (ADUs) as one method to combat surging housing prices. Use of ADUs is an effective means of increasing housing density without replacing single family housing zones with new multi family residential construction. Additionally, ADUs are often designed to be rented, generating supplementary income for the homeowner. Ten detached ADU (or DADU) designs are pre-approved by the City of Seattle and are available online to entice homeowners. However, a 2019 city survey shows that there are calls for an increased focus on sustainability and cost in the construction of ADUs\cite{seattlePreapprovedPlansAccessory2019}. These out of the box designs do not offer the scalability or energy efficiency that an ADU designed specifically per site offers.

%TODO Expand reason for why this tool is necessary to drive ADU demand

Homeowner’s associations and other local organizations have historically fought back against any proposed density increases through zoning or other method. However, Seattle and the other aforementioned cities in the Pacific Northwest have succeeded in allowing for the construction of ADUs in recent decades. The pre-approved designs are free, but require payment of around one thousand dollars [check this] for approval and come with one significant downside of many other pre-designed structures- a lack of contextual design and individualization.

This research explores whether the use of genetic algorithms via shape grammar methodology can effectively optimize DADU plans to site context, increases building performance or further encourages construction. The proposed methodology begins by reading example site data from the city of Seattle including building and vegetation context from the Seattle GIS. Next, a genetic algorithm explores the design space for a viable floor plan solution based on a fitness function. This fitness function evaluates individual designs according to predefined traits. Traits to evaluate include window to wall ratio, insulation depth/type, ventilation strategy, and shading technique. Locating the correct combination of traits to minimize (or maximize) which results in a higher performance DADU is the desired outcome. Resulting designs will be analyzed and compared via energy performance simulation. The end-goal is to develop a computational tool using the aforementioned system to conduct automated site analysis, as well as parametric generation of DADUs.

\section{Literature Review}
%TODO include feedback info on information from seattle city ADU survey- see obsidian notes: city survey respondents said “»  **Successful designs respond to site and context.** Some respondents lauded design choices that reflect a site’s opportunities and constraints or the architecture and climate of the Pacific Northwest. “also: “**Various types of spaces are valuable.** Respondents expressed interest in designs that include patios, rooftop decks, storage, and garages. “ - **SPACES TO CONSIDER FOR ALGORITHM** also: **DIY opportunities.** Respondents encouraged designs that allow homeowners to do a substantial amount of the construction themselves. “ - **SELF BUILD//NO ARCHITECT/BUILDER INVOLVEMENT don’t say no to this yet!!! (family/friend labor?)**also: **Flexibility.** Respondents emphasized the desirability of allowing mix-and-match options, such as multiple roof options, that make the design somewhat customizable. “ - **SHAPE GRAMMAR?**

\section{Methods} 
\subsection{Design constraints and decisions}
%TODO code restrictions: height limitations, lot size, etc via aduniverse-seattlecitygis.hub.arcgis.com/pages/code
%TODO touch on how 3 varying sites were chosen in 3 neighborhoods: central district, first hill and madrona
%TODO include table of R values for each layer, as well as U value of wall assembly, description of wood frame construction chosen, and why chosen.
\subsection{Energy simulation and analysis}
\subsection{Genetic algorithm}

\section{Results}

\section{Conclusion}

\subsection{Future work}
Benefit of a pure python tool 
Looking forward, moving on from the Grasshopper platform gives the benefit of non-reliance on developers to maintain the software in which the tool depends. Creation of such a tool in a singular programming language (in this case Python) further offers the ability to quickly and easily run the tool on a high variance of devices. In turn, this theoretically increases the rate of adoption by lowering the requirement to use the design tool. Additionally, Python is used within Rhino/Grasshopper (or a flavor thereof), provides many useful math and geometry libraries, and has options for injection into a web app (Flask and Django).
Limitations of Galapagos/Grasshopper workflow
Utilizing Galapagos and Grasshopper offers a ‘sandbox’ environment to begin to understand genetic algorithms and to arrive at tangible design solutions faster than using a home-brewed algorithm. Galapagos offers many fewer input parameters and a much narrower scope in which to define the fitness function. In its out-of-the-box form, Galapagos only accepts number sliders as input and can only optimize integers and floats- in reality, there is not means in which to define a true fitness function, only numerical values to target. 


Figures are called in as follows:
\begin{verbatim}
\begin{figure}
\centering
\includegraphics{<figure name>}
\caption{<Figure caption>}
\end{figure}
\end{verbatim}

%\begin{figure}
%\setlength{\fboxsep}{0pt}%
%\setlength{\fboxrule}{0pt}%
%\begin{center}
%\begin{boxedverbatim}
%\begin{table}
%\small\sf\centering
%\caption{<Table caption.>}
%\begin{tabular}{<table alignment>}
%\toprule
%<column headings>\\
%\midrule
%<table entries
%(separated by & as usual)>\\
%<table entries>\\
%.
%.
%.\\
%\bottomrule
%\end{tabular}
%\end{table}
%\end{boxedverbatim}
%\end{center}
%\caption{Example table layout.\label{F2}}
%\end{figure}

%\subsection{Cross-referencing}
%The use of the \LaTeX\ cross-reference system
%for figures, tables, equations, etc., is encouraged
%(using \verb"\ref{<name>}" and \verb"\label{<name>}").
%
%\subsection{Endnotes}
%Most \textit{SAGE} journals use endnotes rather than footnotes, so any notes should be coded as \verb+\endnote{<Text>}+.
%Place the command \verb+\theendnotes+ just above the Reference section to typeset the endnotes.
%
%To avoid any confusion for papers that use Vancouver style references,  footnotes/endnotes should be edited into the text.

\section{Copyright statement}
Please  be  aware that the use of  this \LaTeXe\ class file is
governed by the following conditions.

\subsection{Copyright}
Copyright \copyright\ \volumeyear\ SAGE Publications Ltd,
1 Oliver's Yard, 55 City Road, London, EC1Y~1SP, UK. All
rights reserved.

\subsection{Rules of use}
This class file is made available for use by authors who wish to
prepare an article for publication in a \textit{SAGE Publications} journal.
The user may not exploit any
part of the class file commercially.

This class file is provided on an \textit{as is}  basis, without
warranties of any kind, either express or implied, including but
not limited to warranties of title, or implied  warranties of
merchantability or fitness for a particular purpose. There will
be no duty on the author[s] of the software or SAGE Publications Ltd
to correct any errors or defects in the software. Any
statutory  rights you may have remain unaffected by your
acceptance of these rules of use.

\begin{acks}
This class file was developed by Sunrise Setting Ltd,
Brixham, Devon, UK.\\
Website: \url{http://www.sunrise-setting.co.uk}
\end{acks}

\begin{funding}
This research received no specific grant from any funding agency in the public, commercial, or not-for-profit sectors.
\end{funding}

\begin{dci}
	The author declares that there is no conflict of interest.
\end{dci}

\bibliographystyle{SageV}
\bibliography{597_bibliography}

%\begin{thebibliography}{99}
%\bibitem[Kopka and Daly(2003)]{R1}
%Kopka~H and Daly~PW (2003) \textit{A Guide to \LaTeX}, 4th~edn.
%Addison-Wesley.
%
%\bibitem[Lamport(1994)]{R2}
%Lamport~L (1994) \textit{\LaTeX: a Document Preparation System},
%2nd~edn. Addison-Wesley.
%
%\bibitem[Mittelbach and Goossens(2004)]{R3}
%Mittelbach~F and Goossens~M (2004) \textit{The \LaTeX\ Companion},
%2nd~edn. Addison-Wesley.
%
%\end{thebibliography}

\end{document}
