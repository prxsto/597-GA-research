\chapter{Conclusion}
\label{chap:conclusion}

Benefit of a pure python tool

Looking forward, moving on from the Grasshopper platform gives the benefit of non-reliance on developers to maintain the software in which the tool depends. Creation of such a tool in a singular programming language (in this case Python) further offers the ability to quickly and easily run the tool on a high variance of devices. In turn, this theoretically increases the rate of adoption by lowering the requirement to use the design tool. Additionally, Python is used within Rhino/Grasshopper (or a flavor thereof), provides many useful math and geometry libraries, and has options for injection into a web app (Flask and Django).

Limitations of Galapagos/Grasshopper workflow

Utilizing Galapagos and Grasshopper offers a ‘sandbox’ environment to begin to understand genetic algorithms and to arrive at tangible design solutions faster than using a home-brewed algorithm. Galapagos offers many fewer input parameters and a much narrower scope in which to define the fitness function. In its out-of-the-box form, Galapagos only accepts number sliders as input and can only optimize integers and floats- in reality, there is not means in which to define a true fitness function, only numerical values to target. 
