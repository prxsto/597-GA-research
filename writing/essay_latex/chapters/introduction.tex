\chapter{Introduction}
\label{chap:intro}

The United States is amid an unprecedented housing crisis, stemming from issues outside of housing, such as neoliberal cuts to social benefits spending and the increasing privatization of essentially every aspect of American life. The result is a rising population of unhoused peoples and the inability for many to become a homeowner or to even afford minimum monthly rent in many cities. Solutions to this dilemma are neither straightforward nor definite. However, as designers and architects, this problem can only be addressed at the symptom-level, while advocating at the root cause. 

%TODO Expand background portion of intro

Seattle, Washington; Portland, Oregon; and Vancouver, British Columbia are exploring the use of accessory dwelling units (ADUs) as one method to combat surging housing prices. Use of ADUs is an effective means of increasing housing density without replacing single family housing zones with new multi family residential construction. Additionally, ADUs are often designed to be rented, generating supplementary income for the homeowner. Ten detached ADU (or DADU) designs are pre-approved by the City of Seattle and are available online to entice homeowners. However, a 2019 city survey shows that there are calls for an increased focus on sustainability and cost in the construction of ADUs\cite{seattlePreapprovedPlansAccessory2019}. These out of the box designs do not offer the scalability or energy efficiency that an ADU designed specifically per site offers.

%TODO Expand reason for why this tool is necessary to drive ADU demand

Homeowner’s associations and other local organizations have historically fought back against any proposed density increases through zoning or other method. However, Seattle and the other aforementioned cities in the Pacific Northwest have succeeded in allowing for the construction of ADUs in recent decades. The pre-approved designs are free, but require payment of around one thousand dollars [check this] for approval and come with one significant downside of many other pre-designed structures- a lack of contextual design and individualization.

This research explores whether the use of genetic algorithms via shape grammar methodology can effectively optimize DADU plans to site context, increases building performance or further encourages construction. The proposed methodology begins by reading example site data from the city of Seattle including building and vegetation context from the Seattle GIS. Next, a genetic algorithm explores the design space for a viable floor plan solution based on a fitness function. This fitness function evaluates individual designs according to predefined traits. Traits to evaluate include window to wall ratio, insulation depth/type, ventilation strategy, and shading technique. Locating the correct combination of traits to minimize (or maximize) which results in a higher performance DADU is the desired outcome. Resulting designs will be analyzed and compared via energy performance simulation. The end-goal is to develop a computational tool using the aforementioned system to conduct automated site analysis, as well as parametric generation of DADUs.